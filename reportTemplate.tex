\documentclass[12pt]{exam}
\usepackage[utf8]{inputenc}

\usepackage[margin=1in]{geometry}
\usepackage{amsmath,amssymb}
\usepackage{multicol}

\newcommand{\reportName}{KRACK vulnerability report}
\newcommand{\examdate}{\today}

\rfoot{Page \thepage}
\lfoot{KRACK was discovered by Mathy Vanhoef. This report was generated by KRACK+ a tool by Fredrik Walløe, Lars Trinborgholen and Lars Mæhlum}

\pagestyle{head}
\runningheader{\class}{\examnum\ - Page \thepage\ of \numpages}{\examdate}
\runningheadrule



\begin{document}

\noindent
\begin{tabular*}{\textwidth}{l @{\extracolsep{\fill}} r @{\extracolsep{6pt}} l}
\textbf{\reportName} & \examdate
\end{tabular*}\\
\rule[2ex]{\textwidth}{2pt}


This report contains the results of a vulnerability scan meant to determine whether devices are vulnerable to Key Reintallation Attacks. Note that the operating system is a best guess that may be inaccurate.\newline
FORKLARING AV ANGREPET I SVÆRT KORTE TREKK: type hva vil det si at man er sårbar mot groupkey eller pairwise. Bør være en ikke-teknisk forklaring med fokus på hva som er farene.  \newline

% Results of scan are added by the parse script
\noindent \textbf{Vulnerable devices:} \newline \newline
Device: \newline
Operating system: Android 5.0 - 6.0 (Linux 3.10)
Type: phone \newline
MAC: \newline
%Devices that are not vulnerable should get the following line: "Device does not appear to be vulnerable to KRACK.
Vulnerable to: \newline
Group Key Reinstallation Attacks
Pairwise Key Reinstallation Attacks
\end{document}
54:27:58:63:14:aa




