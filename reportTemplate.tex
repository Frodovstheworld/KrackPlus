\documentclass[12pt]{exam}
\usepackage[utf8]{inputenc}

\usepackage[margin=1in]{geometry}
\usepackage{amsmath,amssymb}
\usepackage{multicol}

\newcommand{\reportName}{KRACK vulnerability report}
\newcommand{\examdate}{\today}

\rfoot{Page \thepage}
\lfoot{KRACK was discovered by Mathy Vanhoef. This report was generated by KRACK+ a tool by Fredrik Walløe, Lars Trinborgholen and Lars Mæhlum}

\pagestyle{head}
\runningheader{\class}{\examnum\ - Page \thepage\ of \numpages}{\examdate}
\runningheadrule



\begin{document}

\noindent
\begin{tabular*}{\textwidth}{l @{\extracolsep{\fill}} r @{\extracolsep{6pt}} l}
\textbf{\reportName} & \examdate
\end{tabular*}\\
\rule[2ex]{\textwidth}{2pt}


This report contains the results of a vulnerability scan meant to determine whether devices are vulnerable to Key Reintallation Attacks. Note that the operating system is a best guess that may be inaccurate.\newline
FORKLARING AV ANGREPET I SVÆRT KORTE TREKK: type hva vil det si at man er sårbar mot groupkey eller pairwise. Bør være en ikke-teknisk forklaring med fokus på hva som er farene.  \newline

% Results of scan are added by the parse script
\noindent \textbf{Vulnerable devices:} \newline \newline

% For each scanned device, this is the format:
%Device nr. <x> \newline
%MAC: <mac> \newline
%IP: <IP> \newline
%\newline
% (If the device is vulnerable against at least one of the two types):
% Vulnerable to: \newline
% (if the device is vulnerable to pairwise):
% Pairwise Key Reinstallation Attacks \newline
% (if the device is vulnerable to group key reinstallation attacks):
% Group Key Reinstallation Attacks \newline
% (If the device is not vulnerable at all):
% Not vulnerable \newline
% \newline \newline

% Should start right below here
Device nr. 1\newline
Mac: 4444.aaaa.1111.2222\newline
IP: 192.168.1.3\newline
\newline
Vulnerable to\newline
Group Key Reinstallation Attacks\newline
